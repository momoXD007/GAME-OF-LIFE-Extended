% \documentclass[10pt]{scrartcl}
\documentclass[10pt,twocolumn]{scrartcl}

\usepackage[utf8]{inputenc}
\usepackage[T1]{fontenc}
\usepackage[ngerman]{babel}

\usepackage{amsmath}
\usepackage{amssymb}

\usepackage{graphicx}
\usepackage{tabularx}

\setlength{\parindent}{0cm}
\setlength{\parskip}{3mm}
\setlength{\textheight}{23.8cm}
\setlength{\headheight}{1cm}
\setlength{\topmargin}{-10mm}

\setlength{\oddsidemargin}{0cm}
\setlength{\evensidemargin}{0cm}
\setlength{\textwidth}{16cm}
\setlength{\columnsep}{8mm}

\usepackage{multicol}
\usepackage{colortbl}
\usepackage{xcolor} 
\usepackage[hyphens]{url}
\definecolor{grau}{gray}{0.95}
\definecolor{dunkelgrau}{gray}{0.85}

\usepackage[normal]{caption}
\usepackage{lipsum}

\setlength{\parindent}{5mm}
\setlength{\parskip}{0mm}

\usepackage{float}
\restylefloat{figure}

\renewcommand{\topfraction}{0.75}
\renewcommand{\textfraction}{0.2}

%###########################################################
% die Sachen mit der Kopfzeile
\usepackage{lastpage}
\usepackage{fancyhdr}
\fancyhf{} % leere alle Felder
\fancyhead[R]{\footnotesize Marvin Klose, Michael Josef Wieneke\\ Inga Miadowicz}
\fancyhead[L]{\footnotesize Ausgewählte Methoden der
Datenanalyse, \\ Modellierung und Simulation} % Titel des Aufsatzes
\fancyfoot[C]{\footnotesize \thepage/\pageref{LastPage}}
% \fancyfoot[C]{\footnotesize \thepage}
\renewcommand{\headrulewidth}{0.4pt} % obere Trennlinie
\pagestyle{fancy}
%###########################################################

\newcommand{\ownsection}[1]{\begin{center}\LARGE\bf#1\end{center}}

\begin{document}

\twocolumn[
\ownsection{Ausbreitung von Infektionskrankheiten innerhalb einer Population}

\begin{center}
Inga Miadowicz (Inga.Miadowicz@gmx.de) \\
Marvin Klose (Marvin.Klose@sap.com) \\
Michael Josef Wieneke (Michael.Wieneke@sap.com)\\
Mannheim, November 2014
\end{center}
\vspace*{5mm}
]

\section*{Abstract}
Jährlich sterben Millionen von Menschen an den verschiedensten Krankheiten. Sie infizieren dabei sich durch Übertragungswege wie Tröpfchen, Kontakt, Lebensmittel oder Trinkwasser. Medikamente und Therapien gibt es für fast alle bekannten Krankheiten, dennoch gibt ist Krankheiten mit tödlichem Ausgang. Deshalb ist interessant zu beobachten, wie sich Krankheiten unter bestimmten Parametern ausbreiten und wie Faktoren wie Resistenzen den Verlauf verändern können. Um schon möglichst präventiv Maßnahmen treffen zu können, damit erst gar keine Pandemie ausbricht und die Zahl möglicher Todesopfer zu minimieren. Der aktuellste Fall einer solchen akuten Bedrohung ist Ebola, welches momentan noch hauptsächlich in Westafrika auftritt. Dieses Themengebiet is nicht nur in der Biologie relevant, sondern lässt auch auf die Informatik übertragen. Computer können sich über Netzwerke und das Internet ebenfalls mit solche Viren oder Würmern anstecken. Dies ist jedoch nicht Bestand der Ausarbeitung.

\section*{Einleitung}

Hier können Sie den Leser auf eine kleine Exkursion
vom Großen ins Kleine bis hin zu Ihrer Problemstellung mitnehmen.
Die großen Fragen, wie 'Woher kommen wir?', 'Was machen wir?' 
oder 'Und warum ist es so spannend sich mit dem zu 
beschäftigen, was Sie im folgenden lesen werden?' können Sie
hier behandeln.

Am Ende kann ein wenig zur Gliederung gesagt werden,
damit sich der Leser ein Bild vom Ganzen machen kann.

Achja, wenn man in einem Forschungsfeld unterwegs ist, dann
werden Sie womöglich ähnliche Einleitungen in verschiedenen
Artikeln finden. Böse gesagt, kennst Du eine 
Einleitung, kennst Du fast alle, was natürlich nicht
ganz wahr ist. Wenn man aber neu in einem Gebiet
ist, dann erhält man in der Einleitung einen schnellen Überblick.
Meistens finden sich die wichtigsten und aktuellsten 
Fakten und Veröffentlichungen des Forschungsgebietes in der Einleitung.
\subsection*{Bestehende Modelle}

SI SIR SIRS... Inga? 


\section{Vergleich mit dem SIR Modell}
Da die mathematische Beschreibung, sowie das Programm beide den Verlauf von Krankheiten simulieren, liegt ein Vergleich der beiden Herangehensweisen nahe. 
Dieser soll anhand eines konkreten Testszenarios angewandt werden, bei dem beide Modelle mit denselben Daten eine Prognose über die Entwicklung einer Krankheit liefern sollen.\\
Der Vergleich soll eine Einschätzung der Repräsentativität, des dem Projekt zugrunde liegenden Programms ermöglichen und Schwächen und Stärken der beiden Modelle herausstellen.\\
Dazu wird zunächst das Testszenario vorgestellt und dann in beiden Modellen angewandt.
Auf einer kurze Vorstellung der Ergebnisse in den Modellen folgt zu guter Letzt eine GegenÜBERSTELLUNG MUHUHUHU der beiden Resultate.
Für die Visualisierung der Daten des SIR-Modells wurde im Nachfolgenden die SIR-Simulation von Hans Nesse benutzt.%\cite{XXX}

\subparagraph{Szenario}
Das Testszenario spielt in einem kleinen Dorf mit 10150 Einwohnern zur Zeit des Mittelalters.\\
Damals hat die Pest, oft bezeichnet als "Schwarzer Tod" von 1347 bis 1353 ein Drittel des europäischen Bevölkerung das Leben gekostet. Die Diagnose Pest war damals gleichzusetzen mit dem Todesurteil, da nur ein sehr geringer Teil der Bevölkerung diese Krankheit mit den medizinischen Stand überleben konnte. Während einige Landstriche komplett entvölkert wurden blieben einige Gebiete verschont. In Italien starb fast jeder 5. an der Krankheit, in Deutschland hingegen jeder 10.\\
Aufgrund der regionalen Unterschiede und der wenigen verlässlichen Daten ist es besonders schwer die die Kennzahlen von Krankheiten möglichst Realitätsnah wiederzugeben, insbesondere wenn die Krankheit tödlich ist.\\
Für den Vergleich wird angenommen, dass die Krankheit eine Infektionsrate von 33\% hat, da die Heilungschancen sehr gering sind und ein Drittel an der tödlichen Krankheit gestorben ist.\cite{} Zudem stirbt ein erkranktes Individuum pro Runde zu 70\% an seiner Krankheit. Die Wahrscheinlichkeit zu genesen oder sogar Resistent zu werden liegt jeweils nur bei 2\%.\cite{}\\
Hier muss zwischen den Eingabedaten für das SIR-Modell und denen des Programms unterschieden werden:

\section*{Ergebnisse}
Grundsätzlich lässt sich sagen, dass die Pest in beiden Modellen komplett ausstirbt. Beim SIR Modell liegt dies trivialerweise daran, dass die sich die gesamte Population im R-Zustand befindet. Betrachtet man nun das vorgestellte Modell lassen sich so leicht keine Aussagen treffen, weil das vorgestellt Modell nicht deterministisch ist, deshalb lassen sich bei nur einmaliger Ausführung nicht wirklich repräsentative Ergebnisse ermitteln. Aus diesem Grund wurde tausend mal mit den selben Parametern simuliert, um eine Abschätzung der Ergebnisse zu ermöglichen. Die Ergebnisse zu den zu erwartenden Werten mit den entsprechenden Zuständen, entweder Tod oder Lebendig, befinden sich in der Tabelle \ref{tab:ergebnisse}.
\begin{table}[H]
	\caption{Ergebnisse der Pestsimulation}
	\label{tab:ergebnisse}
	\centering
	\begin{tabular}{|c|ccc|}
		\rowcolor{dunkelgrau}
		\hline
		Zustand &	Minimum & Maximum & Mittelwert \\ \hline
		Tod & 7500 & 8200 & 7864 \\ 
	\rowcolor{grau}		Lebendig & 3050 & 3750 & 3386\\
	\hline
	\end{tabular}
\end{table}
Wir stellen fest, dass bei einer Ausgangsgröße der Population von 10150 Individuen sich die Schwankung nur auf 6,89\% beläuft. Damit unterscheiden sich die Ergebnisse jedoch deutlich von denen des SIR Modells, bei dem die komplette Population nach den 50 Zeitschritten im R-Zustand befindet. Die maximale Anzahl an Infizierten erreichen SIR und das vorgestellte Modell in einem ähnlichen Zeitraum. Beim SIR Modell tritt das Maximum der Infizierten nach etwa 2,5 Tagen auf und unser Modell erreicht im Schnitt das Maximum der Infizierten nach 3,1 Tagen. Das ist nur etwa einen halben Tag später im Schnitt.
Trotz alledem ist eine interessante Bemerkung, dass nicht alle Individuen aussterben. Dies kann verschiedene Gründe haben. Zum einen besteht bei dem Modell die Chance, dass die Zellen wieder in den gesunden Zustand zurückkehren. Eine weitere Möglichkeit ist, dass die Krankheit verhältnismäßig schnell zum Tod führt und so längst nicht alle Zellen infiziert werden. Durch eine günstige Lage kommen sie demnach nicht mit Infizierten Zellen in Kontakt.   

\section{Diskussion}
Aus den Ergebnissen lassen sich nun gewisse Punkte identifizieren, die die beiden Modelle unterscheidet.
Angefangen bei der Verlässlichkeit und Reproduzierbarkeit, dass SIR Modell als mathematisches Modell ist statisch und verhält sich unter gegebenen Parametern immer gleich. Es gibt keine Schwankung bei den Werten. Währenddessen es beim vorgestellten Modell nicht wirklich vorhersagbar ist, wie die Simulation genau ausgeht. Lediglich lässt sich bei hinreichend großer Anzahl von Iterationen ein Mittelwert der simulierten Werte ermitteln. Die Werte sind jedoch nicht deterministisch.
Unser Modell liegt damit eventuell etwas näher an der Realität, da man auch im echten Leben nie eine Entwicklung einer Epidemie mit 100\% Wahrscheinlichkeit voraussagen kann. Denn die Parameter können sich im Verlauf der Zeit verändern. Vorstellbar wäre es, dass die Viren Resistenzen gegen bekannte Medikamente bilden oder gar mutieren.
Ein weiterer interessanter Fall ist, wenn eine Krankheit nur in einer bestimmten Region ausbricht und von dort aus auch nicht in andere Gebiete vordringen kann, weil die betroffene Population schnell ausstirbt oder keine bis kaum Kontakte nach Außen hat. 
Dieser Punkt offenbart eine Stärke des vorgestellten Modells. Das Modell ermittelt Zustandsübergänge durch die unmittelbare Nachbarschaft der Zelle und nicht über den gesamten Anteil der Population.
Dadurch können Zellen einen Vorteil haben, deren Nachbarschaft kaum bis gar nicht besiedelt ist.
Ein weiterer Vorteil unserer Modells ist, dass Wirtsbeziehungen abgebildet werden. Dies kann unter Umständen für bestimmte Krankheiten die nur von Tieren von auf Menschen übertragen werden können. Ein solches Beispiel ist das auch in Deutschland immer häufiger vorkommende Denguefieber.
Auch bietet unsere Modell eine grafische Repräsentation, so dass man anschaulich sieht, wie die Population sich entwickelt.


Jedoch gibt es auch einige Punkte der Realität die unser Modell nicht abbilden kann. Inkubationszeiten wie sie in der Realität tatsächlich vorkommen werden nicht berücksichtigt. Die Population verhält sich ohne Krankheit auch statisch, das heißt es gibt keine Alterung der Zellen und auch keine Reproduktion. Ebenso sind alle Zellen gleich anfällig für Krankheiten, das spiegelt die Realität nicht unbedingt wieder, da ältere Menschen meist schwächer und somit anfälliger für Krankheiten als jüngere sind.
Eine weitere Schwäche ist es, dass die Zellen die resistent geworden sind, einfach auf dem Raster weiter existieren. Für diese Zellen bedeutet die Resistenz somit das ewige Leben, da bisher keine Alterung der Zellen berücksichtigt wird. 

Insgesamt ist das Modell dementsprechend etwas dynamischer als das SIR Modell. Obwohl es nicht deterministisch ist scheint es jedoch etwas näher an der Realität zu sein. Für wichtige Entscheidungen im Katastrophenschutz kann andererseits genau dieses auch ein Negativpunkt sein, da man nur von verschiedenen Szenerien ausgehen kann und nicht von konkreten absoluten Zahlen.

\section*{Ausblick}
Da die Welt bisher nur aus einem bloßen Raster mit den Koordinaten $x$ und $y$ besteht, wäre ein erster Schritt sich der Realität weiter anzunähern, dass die Welt keine harte Ecken hat. Wie sie noch in Abbildung \ref{fig:welt} zu erkennen sind.
\begin{figure}[H]
	\centering
	\includegraphics[width= 0.45\textwidth]{./images/map.png}
	\caption{Bisheriges einfaches Raster}
	\label{fig:welt}
\end{figure}
Um dieses Problem zu lösen ist die Idee, dass Raster in Form einer Kugel zu modellieren, so dass die Übergänge weicher werden. Des Weiteren erinnert diese Form eher an die der Erde. Für die Simulationslogik sähe das Raster als Folge dessen wie in Abbildung \ref{fig:kugel} gezeigt aus. Ebenso wären natürliche Hindernisse wie Meere oder ähnliches in der Modellierung denkbar.
\begin{figure}[H]
	\centering
	\includegraphics[width= 0.3\textwidth]{./images/kugel.pdf}
	\caption{Raster als Kugel\cite{WikiKugel:2014}}
	\label{fig:kugel}
\end{figure}


Unabhängig davon besteht auch bei der Simulationslogik selbst Verbesserungspotential. Denn ein mögliches Ziel wäre es die Realität bestmöglich auch im Modell abzubilden.
Die Umsetzung von Inkubationszeiten trüge bereits einen enormen Anteil zu diesem Ziel bei. 
Das Problem der Inkubationszeiten lässt sich lösen, indem die Krankheiten weiter spezifiziert werden. Der Zustand der Zellen verändert sich demnach nicht direkt, sondern in Abhängigkeit der Inkubationszeit die bei der Krankheit zu erwarten ist.

Die Zellen, die schon als Objekte existieren, könnten durch weitere Attribute sinnvoll ergänzt werden, damit sie die Realität besser abbilden.
Vor allem Alterung spielt hier ein zentrale Rolle. Das junge widerstandsfähiger als ältere Zellen sind gilt für die meisten Krankheiten, jedoch nicht unbedingt für alle und hängt auch von weiteren Faktoren ab.Das je nachdem bestimmte Zellen anfälliger sind berücksichtigt unser Modell jedoch noch nicht. Ein Lösungsansatz für diese Problem wäre zum einem ein Alter für die Zelle einzuführen als auch die Krankheit mit verschiedenen Wahrscheinlichkeiten für die jeweilige Zielgruppe zu modellieren. 
Durch die Alterung ist  ebenfalls das Problem gelöst, dass Zellen theoretisch ewig leben können, wenn sie keiner Krankheit zum Opfer fallen. 

Dies gilt allerdings nur, wenn zusätzlich eine weitere Voraussetzung erfüllt ist. Die Simulation benötigt eine gewisse Lebenserwartung der Zellen. Diese kann je nach Region oder bisherigen Iterationen variieren und sorgt dafür, dass Zellen nicht ewig leben können.
Zeitgleich wirft diese Lösung ein neues Problem in den Raum und zwar, dass es keine Reproduktion der Zellen gibt. Die Population stürbe auf lange Sicht gesehen aus. Ein möglicher Lösungsansatz wäre hier, mit jeder Iteration eine gewisse Anzahl von neuen Zellen zu erzeugen. Allerdings ist diese Berechnung keineswegs trivial, weil verschiedene Faktoren wie Geschlechterverhältnis, Krankheitsübertragungen während der Schwangerschaft, Überbevölkerung und weitere diese beeinflussen.

Des Weiteren verwendet die Simulation bisher nur die unmittelbare Nachbarschaft. Betrachtet man nun Ballungsgebiete wie New York, Peking oder Tokio ist unmittelbare Nachbarschaft nicht unbedingt ausreichend. Die Wahrscheinlichkeit mit einer Infizierten Person in Kontakt zukommen ist größer als in ländlichen Regionen. Deshalb wäre eine Lösung die Nachbarschaft auszudehnen, so dass weitere Zellen für die Zustände verantwortlich sind. Beispielsweise wie in Abbildung \ref{fig:ewnachbar} gezeigt. Die Moore-Nachbarschaft wird nur auf den Radius 2 erhöht. Die gesunde Zelle in der Mitte hat nun 2 Infizierte Zellen, die für die Berechnung des neuen Zustandes relevant sind. 


\begin{figure}[H]
	\centering
	\includegraphics[width= 0.3\textwidth]{./images/ewNachbar.pdf}
	\caption{Augmentierte Nachbarschaft}
	\label{fig:ewnachbar}
\end{figure}
An dieser Stelle könnte man zusätzlich eine Art Gauß-Filter verwenden, um auch die Distanz bei der Gewichtung mit einfließen zu lassen. Die Zellen im unmittelbaren Umkreis hätten somit einen größeren Einfluss als diejenigen, die weiter entfernt liegen.
Dafür könnte man 
\begin{center}
 $A = 
\begin{bmatrix}
b & b & b & b & b \\
b & a & a & a & b \\
b & a & 0 & a & b \\
b & a & a & a & b \\
b & b & b & b & b \\
\end{bmatrix}
$ als Faltungsmatrix
\end{center} verwenden, um die Distanz mit Faktoren $a,b$ zu gewichten. Beispielsweise könnte man die unmittelbare Nachbarschaft also $a=1$ und die Nachbarschaft mit dem Radius 2 mit $b=\frac{2}{3}$ bewerten. Ebenso könnte man die Diagonale aufgrund höherer Distanz zusätzlich geringer gewichten.

In der vorliegenden Version bewegen sich alle Zellen vollkommen zufällig. Da aber Menschen als auch Tiere sich meist zielgerichtet bewegen, wäre es realistischer anzunehmen dass Zellen einen Zielpunkt auf dem Raster haben oder sich zumindest nicht nach einem Schritt wieder direkt zu ihrer alten Position zurück bewegen. Hier bietet sich beispielsweise das sogenannte \glqq{}Correlated random walks\grqq{}. Bei dem davon ausgegangen wird dass eine Reihe von aufeinanderfolgenden Bewgungen tendziell immer in die Gleiche Richtung geht der Einfluss der vorherigen Bewegungen aber immer weiter abnimmt umso länger sie zurückliegen\cite{Codling:2008}.

Eine weitere Schwachstelle der Implementation ist die Verwendung eines Generators von Zufallszahlen. Dieser kann nur pseudo-Zufallszahlen erzeugen welche von der Uhrzeit der Initialisierung in Nanosekunden und der Aufrufe des Generators abhängen\cite{Oracle:2014}. Sollte es also vorkommen das zwei Simulationen zur exakt gleichen Uhrzeit mit den komplett gleichen Parametern gestartet werden, werden sie identische Ergebnisse liefern.\\
Dies könnte man verhindern indem man echte Zufallszahlen aus natürlichen zufälligen Ereignissen ermittelt. Dies tut zum Beispiel die Website \url{http://www.random.org}, diese ermittelt Zufallszahlen aus dem Hintergrundrauschen der Atmosphäre. Eine Einbindung dieses oder eines ähnlichen Services würde obiges beseitigen. Allerdings ist es fraglich ob es bei der Seltenheit bei der das Problem auftritt, überhaupt notwendig ist, sich mit dieser Problematik weiter zu befassen.



\section*{Danksagung}
Wir möchten uns an dieser Stelle bei \emph{Dr. Gerhards} bedanken. Einerseits für die fachliche Hilfe bei diesem Projekt und andererseits für seine hilfreichen Anmerkungen nach der Präsentation im Plenum, die die Basis dieser Ausarbeitung darstellt.\\
Des Weiteren möchten wir uns bei unseren Ausbildungsbetrieben der \emph{SAP SE} und der \emph{ALDI GmbH \& Co. oHG} ohne die es uns erst gar nicht möglich gewesen wäre an Dr. Gerhards Vorlesung teilzunehmen.



\begin{thebibliography}{99}
\bibitem{gehlhoff2007chronik}
B. Gehlhoff,  \textit{Chronik Jahresrückblick 2006}, 2007,
\url{http://books.google.de/books?id=JtPnf5K1CaIC},
Chronik-Verlag, Seite 15


\bibitem{Spon:2014}Spiegel Online {\it Die Chronologie der BSE-Krise}, 2000, \url{
	http://www.spiegel.de/politik/ausland/rinderseuche-die-chronologie-der-bse-krise-a-105210.html
}, Einsichtnahmen 22.11.2014 10:24


\bibitem{welt:2014}Die Welt {\it Warum die Spanische Grippe so verheerend war} , 2014, \url{http://www.welt.de/gesundheit/article127418306}, Einsichtnahmen 22.11.2014 17:10

	
\bibitem{UNAIDS:2014}UNAIDS {\it FACT SHEET 2014}, 2014,
	\url{http://www.unaids.org/sites/default/files/documents/20141118_FS_WADreport_en.pdf},
	Seite 1, Einsichtnahme 22.11.2014 17:19

	B. Gehlhoff,  \textit{Chronik Jahresrückblick 2006}, 2007,
	\url{http://books.google.de/books?id=JtPnf5K1CaIC},
	Chronik-Verlag, Seite 15



\bibitem{jdmurray}
	J.D. Murray,  \textit{Mathematical Biology: I. An Introduction, Third Edition}, 2002,
	\url{http://www.ift.unesp.br/users/mmenezes/mathbio.pdf},
	Springer-Verlag, Seite 329	


\bibitem{Kesh}
	L. Edelstein-Keshet, \textit{Chronik Jahresrückblick 2006}, 2005,
	\url{http://books.google.de/books?id=uABYP1hnsf0C&printsec=frontcover&hl=de&source=gbs_ge_summary_r&cad=0#v=onepage&q=SIR&f=false},
	SIAM (Society for Industrial and Applied Mathematics), Seite 242-252
	

\bibitem{sebM}
	S. Möhler,  \textit{Ausbreitung von Infektionskrankheiten}, 2004,
	\url{http://www.mathe.tu-freiberg.de/~wegert/Lehre/Seminar3/moehler.pdf}

%Sebastian M¨ohler: Ausbreitung von %Infektionskrankheiten
%Literatur: [2] 242–257, [3] 610–619
	

\bibitem{Weisstein:2014}E. Weisstein {\it Moore Neighborhood}, 2014 \url{http://mathworld.wolfram.com/MooreNeighborhood.html}, Einsichtnahme 22.11.2014 09:14


\bibitem{WikiKugel:2014}Wikipedia {\it Kugel}, \url{http://de.wikipedia.org/wiki/Kugel}, Einsichtnahme 22.11.2014 11:22, mit Änderung übernommen


\bibitem{Codling:2008}E.A. Codling, M.J. Plank und S. Benhamou {\it Random walk models in biology}, 2008; Journal of The Royal Society Interface S. 813-834

\bibitem{Oracle:2014}Oracle  {\it Java Dokumentation: Random},2011; \url{https://docs.oracle.com/javase/6/docs/api/java/util/Random.html}, Einsichtnahme 22.11.2014 19:02
\end{thebibliography}

\end{document}