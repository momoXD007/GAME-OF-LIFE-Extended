\section*{Einleitung}

Die hier vorgestellte Ausarbeitung entstand im Rahmen der Vorlesung \glqq Ausgewählte Methoden der Datenanalyse, Modellierung und Simulation\grqq\; an der Dualen Hochschule Baden-Württemberg.
In der Vorlesung wurde ein Modell entwickelt, mit dem Ziel die Ausbreitung von Infektionskrankheiten zu simulieren.
Zentrale Aspekte sind deshalb die Gesundheit der Bevölkerung und die Veränderungen der Bevölkerungszahl, aufgrund von Epidemien. Somit lässt sich das Thema wissenschaftlich unter dem Begriff der Epidemiologie einordnen.
Dass das Thema nach wie vor Relevanz hat, sieht man an der aktuellen Ebola Epidemie, die sich hauptsächlich in Afrika ausbreitet. Zudem haben auch schon in der Vergangenheit immer wieder Epidemien Aufsehen erregt. Darunter beispielsweise 2006 die Vogelgrippe\cite{gehlhoff2007chronik}
oder 2000/2001 BSE\cite{Spon:2014}. Diese forderten jedoch im Vergleich zur spanischen Grippe, die mehrere Millionen Menschen das Leben kostete, vergleichsweise wenig Todesopfer. Die spanische Grippe war die bisher verheerendste Influenzapandemie und forderte etwa 50 Millionen Todesopfer\cite{welt:2014}. Zusätzlich spricht man bei HIV/AIDS ebenfalls von einer Pandemie bei nach Angaben der Organisation UNAIDS seit Bekanntwerden der Krankheit zwischen 35 und 43 Millionen Menschen gestorben sind\cite{UNAIDS:2014}.
Damit ist ein Ziel des Modells die bestmöglichen Gegenmaßnahmen zu treffen, um die Bevölkerung zu schützen und so die Anzahl der Todesopfer zu minimieren.
Das vorgestellte Modell simuliert dazu die Entwicklung verschiedener Krankheiten innerhalb einer Population über einen gewissen Zeitraum. 
Inhaltlich beginnen wir mit der Erläuterung bestehender Modelle der Wissenschaft. Anschließend folgt die Vorstellung des von uns entwickelten Modells, das wir im Anschluss daran mit SIR-Modell vergleichen wollen. Darauf folgen die Ergebnisse und eine Diskussion. Abschließend gibt es einen Ausblick, der mögliche Erweiterungen unseres Modells behandelt. 

