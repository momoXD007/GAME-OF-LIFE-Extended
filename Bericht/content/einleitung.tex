\section*{Einleitung}

Die hier vorgestellte Ausarbeitung entstand im Rahmen der Vorlesung \glqq Ausgewählte Methoden der Datenanalyse, Modellierung und Simulation\grqq\; an der Dualen Hochschule Baden-Württemberg. Thematisch beschäftigt sich das Modell mit dem Begriff der Epidemiologie, die sich mit der Verbreitung sowie den Ursachen und Folgen von gesundheitsbezogenen Zuständen und Ereignissen in Bevölkerungen oder Populationen beschäftigt. %Zitat Wikipedia
Dass das Thema nach wie vor Relevanz hat, sieht man an der aktuellen Ebola Epidemie. Zudem haben auch schon in der Vergangenheit immer wiederholt Epidemien Aufsehen erregt. Darunter beispielsweise 2006 die Vogelgrippe \cite{gehlhoff2007chronik}
oder 2000/2001 BSE\cite{Spon:2014}. Diese forderten jedoch im Vergleich zur spanischen Grippe, die etwa 50 Millionen Menschen das Leben kostete, vergleichsweise wenig Todesopfer.
Damit ist ein Ziel des Modells die bestmöglichen Gegenmaßnahmen zu treffen, um die Bevölkerung zu schützen und so die Anzahl der Todesopfer zu minimieren.
Das vorgestellte Modell simuliert dazu die Entwicklung verschiedener Krankheiten innerhalb einer Population über einen gewissen Zeitraum. 



%Hier können Sie den Leser auf eine kleine Exkursion
%vom Großen ins Kleine bis hin zu Ihrer Problemstellung mitnehmen.
%Die großen Fragen, wie 'Woher kommen wir?', 'Was machen wir?' 
%oder 'Und warum ist es so spannend sich mit dem zu 
%beschäftigen, was Sie im folgenden lesen werden?' können Sie
%hier behandeln.
%
%Am Ende kann ein wenig zur Gliederung gesagt werden,
%damit sich der Leser ein Bild vom Ganzen machen kann.
%
%Achja, wenn man in einem Forschungsfeld unterwegs ist, dann
%werden Sie womöglich ähnliche Einleitungen in verschiedenen
%Artikeln finden. Böse gesagt, kennst Du eine 
%Einleitung, kennst Du fast alle, was natürlich nicht
%ganz wahr ist. Wenn man aber neu in einem Gebiet
%ist, dann erhält man in der Einleitung einen schnellen Überblick.
%Meistens finden sich die wichtigsten und aktuellsten 
%Fakten und Veröffentlichungen des Forschungsgebietes in der Einleitung.