\section*{Abstract}
Jährlich sterben Millionen von Menschen an den verschiedensten Krankheiten. Sie infizieren dabei sich durch Übertragungswege wie Tröpfchen, Kontakt, Lebensmittel oder Trinkwasser. Medikamente und Therapien gibt es für fast alle bekannten Krankheiten, dennoch existieren Krankheiten mit tödlichem Ausgang. 
Deshalb ist interessant zu beobachten, wie sich Krankheiten unter bestimmten Parametern ausbreiten und wie Faktoren z.B. Resistenzen oder Impfungen den Verlauf verändern können. Um schon möglichst präventiv Maßnahmen treffen zu können, damit erst gar keine Pandemie ausbricht und die Zahl möglicher Todesopfer zu minimieren. Der aktuellste Fall einer solchen akuten Bedrohung ist Ebola, welches momentan noch hauptsächlich in Westafrika auftritt. Dieses Themengebiet ist nicht nur in der Biologie relevant, sondern lässt sich auch auf die Informatik übertragen. Computer können sich über Netzwerke und das Internet ebenfalls mit solche Viren oder Würmern anstecken. Dies ist jedoch nicht Bestand der Ausarbeitung.