%Wie bereits erläutert weißt das SIR-Modell einige Schwächen auf unter Anderem wird in dem Modell davon ausgegangen dass jeder Mensch im Schnitt acht Kontakte zu anderen Menschen hat. In unserem Programm wird dies dynamisch geregelt, dort gibt es einen Raster welchereinem Schachbrett ähnelt auf diesem beliebig großem rechteckigem \glqq{}Spielfeld \grqq{} lassen sich an jeder x,y Position Zellen positionieren. So kann es Zellen geben welche sehr viele Nachbarn haben aber auch Zellen welche gar keinen Nachbar haben. So können auch extreme Situationen realistisch dargestellt werden wie zum Beispiel sehr dünn besiedelte Gegenden oder aber auch mittelalterliche Städte in denen in einem Haushalt durchaus 20 Personen lebten.\\
Verbesserungen gegenüber dem SIR-Modell

\subsection{Bewegung}
\subsection{Wirtsbeziehungen}

\subsection{Bevölkerungsdichte wird berücksichtigt}

\subsection{nicht deterministisch}

\begin{enumerate}
\item{Bewegung}
\item{\emph{Wirtsbeziehungen}



}
\item{\emph{Bevölkerungsdichte wird berücksichtigt}


}
\item{nicht deterministisch}
\end{enumerate}